% !TeX root = RJwrapper.tex
\title{Handling bibliographies with texor}
\author{by Abhishek Ulayil}

\maketitle

\abstract{
This is a small sample article to demonstrate usage of \CRANpkg{rebib} to manage bibliographies.
}

\section{Introduction}
The \CRANpkg{rebib} package is a spin-off from the \CRANpkg{texor} package.

The spin-off cam about because the bibliography section 
in \CRANpkg{texor} package was expanding significantly, enough to deserve a dedicated space.

Thus \CRANpkg{rebib} was born and it has improved over the past few months to include more advanced features and utilities. 

\subsection{What does rebib do?}

In LaTeX, we have multiple ways to declare bibliography entries, which include embedding entries formatted to a given style in the document itself or isolating that section to an external file commonly called a \file{.bbl} file. Then we have the BibTeX format, which is designed to store the full bibliographic information for each reference, making it superior for managing references.

LaTeX has no problem dealing with multiple formats and works well with all these bibliographic formats. However, some web publishing journals like R Markdown articles for the R Journal do not support embedded bibliographies. As a result, they need a BibTeX equivalent of the same.

The \CRANpkg{rebib} package converts embedded LaTeX bibliographies into a close  BibTeX equivalent. This package relieves the author from manual conversion of the documents and aids the project to convert legacy R Journal articles to the current R Markdown format.

\subsection{Features of rebib}

\begin{itemize}
  \item Extracts embedded bibliographic entries from a \file{.tex} file.
  \item Creates the mandatory title and author fields.
  \item Creates the optional URL, ISBN, publisher, pages and year fields, when available.
  \item Stores the remaining information in \verb|"journal"| (internally) and \verb|"publisher"| (when writing BibTeX file).
  \item Ignores commented LaTeX code. 
  \item Tracks citations included in the document.
  \item Aggregates embedded bibliographic entries with references from supplementary \file{.bbl} and \file{.bib} files.
\end{itemize}

\section{Parser}

The magic behind \CRANpkg{rebib} is a parser which parses bibliographic data and sorts them according to the matching regular expressions.

\subsubsection{Stage 1: Read the embedded bibliography block}

This stage is a minor step where it reads the Embedded Bibliography from the LaTeX document. This step also includes filtering out commented lines to avoid reading un-intended entries.

The bibliography is split into entries using the LaTeX macro \verb|\\bibitem| as a marker for a new entry and the resulting data is exported to a variable.


\begin{Sinput}
dir.create(your_article_folder <- file.path(tempdir(), "exampledir"))
example_files <-  system.file("article", package = "rebib")
x <- file.copy(from = example_files,to=your_article_folder,recursive = T)
your_article_path <- paste(your_article_folder,"article",sep="/")
file_name <- rebib:::get_texfile_name(your_article_path)
bib_items <- rebib:::extract_embeded_bib_items(your_article_path,file_name)
bib_items[[1]]
#> [1] "\\bibitem[Ihaka, Ross and Gentleman, Robert]{ihaka:1996}"                          
#> [2] "Ihaka, Ross and Gentleman, Robert"                                                 
#> [3] "\\newblock \\emph{R: A Language for Data Analysis and Graphics.}"                  
#> [4] "\\newblock \\emph{Journal of Computational and Graphical Statistics}, 3:\\penalty0"
#> [5] "299--314, 1996."                                                                   
#> [6] "\\newblock URL : \\url{https://doi.org/10.1080/10618600.1996.10474713}"
\end{Sinput}


\subsubsection{Stage 2: Regex powered parser}


Now, with the chunks of bibliographic entries, each is passed to a parser which will break it down based on regular expressions.
The logic is to use the LaTeX macro \verb|\\newblock| as a placeholder to identify the position of text elements relative to it.

The first value to be parsed is the \verb|unique_id|, also called the citation reference which is used to cite elements inside the article. Usually, this is in the first or second line of the whole entry. The position of the 
\verb|unique_id| will determine the position of the author names.

\begin{Soutput}
bib_items[[1]][1]
#> [1] "\\bibitem[Ihaka, Ross and Gentleman, Robert]{ihaka:1996}"
\end{Soutput}

After reading the \verb|unique_id|, the parser will attempt to read the author name(s).

\begin{Sinput}
bib_items[[1]][2]
#> [1] "Ihaka, Ross and Gentleman, Robert"
\end{Sinput}

Next, the title is extracted based on the position of the new blocks or the end of the bib chunk. 

\begin{Sinput}
bib_items[[1]][3]
#> [1] "\\newblock \\emph{R: A Language for Data Analysis and Graphics.}"
\end{Sinput}

This way the crucial elements of the bibliographic entry (unique\_id, author names and title) are parsed out.
The remaining data is stored as \verb|journal| internally and \verb|publisher| when writing to a new BibTeX file.

\begin{Sinput}
bib_items[[1]][4:6]
#> [1] "\\newblock \\emph{Journal of Computational and Graphical Statistics}, 3:\\penalty0"
#> [2] "299--314, 1996."                                                                   
#> [3] "\\newblock URL : \\url{https://doi.org/10.1080/10618600.1996.10474713}"
\end{Sinput}


A series of filters for ISBN, URL, pages and year fields are applied to search for relevant data from the remaining data. If the data is not available, then it is set as \code{NULL} and will be skipped while writing the BibTeX file. 
There is a lot of filtering of common LaTeX elements and after that, the remaining data is stored in a structured format to be written to a file. 


\begin{Schunk}
\begin{Soutput}
bib_entry <- rebib:::bib_handler(bib_items)
bib_entry
#> $book
#> $book[[1]]
#> $book[[1]]$unique_id
#> [1] "ihaka:1996"
#> 
#> $book[[1]]$author
#> [1] "Ihaka, Ross and Gentleman, Robert"
#> 
#> $book[[1]]$title
#> [1] "R: A Language for Data Analysis and Graphics"
#> 
#> $book[[1]]$journal
#> [1] "Journal of Computational and Graphical Statistics 3:   :"
#> 
#> $book[[1]]$year
#> [1] "1996"
#> 
#> $book[[1]]$URL
#> [1] "https://doi.org/10.1080/10618600.1996.10474713"
#> 
#> $book[[1]]$isbn
#> NULL
#> 
#> $book[[1]]$pages
#> [1] "299--314"
#> 
#> 
#> $book[[2]]
#> $book[[2]]$unique_id
#> [1] "R"
#> 
#> $book[[2]]$author
#> [1] "R Core Team"
#> 
#> $book[[2]]$title
#> [1] "R: A Language and Environment for Statistical Computing"
#> 
#> $book[[2]]$journal
#> [1] "R Foundation for Statistical Computing Vienna Austria    :"
#> 
#> $book[[2]]$year
#> [1] "2016"
#> 
#> $book[[2]]$URL
#> [1] "https://www.R-project.org/"
#> 
#> $book[[2]]$isbn
#> [1] "3-900051-07-0"
#> 
#> $book[[2]]$pages
#> NULL
\end{Soutput}
\end{Schunk}


\subsubsection{Stage 3: BibTeX writer}

After reading the bibliographic entries and splitting out meaningful values from them, we can finally write a structured file in the BibTeX format.

The writer will read the bib chunks one at a time based on the metadata extracted and will write the corresponding data fields. The default entry type is a book, which should not have any problems with the web articles. 
\begin{Schunk}

\begin{Sinput}
file_path <- paste0(your_article_path,"/",file_name)
bib_path <- paste0(your_article_path,"/example.bib")
rebib:::bibtex_writer(bib_entry,file_path)
cat(readLines(paste(your_article_path,"example.bib",sep="/")),sep = "\n")
#> @book{ihaka:1996,
#> author = {{Ihaka, Ross and Gentleman, Robert}},
#> title = {{R: A Language for Data Analysis and Graphics}},
#> publisher = {Journal of Computational and Graphical Statistics 3:   :},
#> pages = {299--314},
#> year = {1996},
#> url = {https://doi.org/10.1080/10618600.1996.10474713}
#> }
#> @book{R,
#> author = {R {Core Team}},
#> title = {{R: A Language and Environment for Statistical Computing}},
#> publisher = {R Foundation for Statistical Computing Vienna Austria    :},
#> year = {2016},
#> url = {https://www.R-project.org/},
#> isbn = {3-900051-07-0}
#> }
\end{Sinput}
\end{Schunk}

Authors who are converting a bibliography from \file{bbl} should check if there are any errors or missing values. 



\section{General usage}

General use of the \CRANpkg{rebib} package is quite easy, requiring only one function call.

\subsubsection{Using with RJournal article}

In bibliography construction mode, \CRANpkg{rebib} will check for existing BibTeX files and will convert the embedded bibliography only when there are no linked BibTeX files in the RJ article.
\begin{verbatim}
%% typically rebib will search for this line in the RJ article, 
%% not the RJwrapper file
\bibliography{example}
\end{verbatim}
If there is no linked BibTeX file, \CRANpkg{rebib} will begin its procedure of converting the embedded bibliography to BibTeX and then link it with the article file as well.
\begin{Schunk}
\begin{Sinput}
# for files without BibTeX source
rebib::handle_bibliography(your_article_path)
#> Warning in get_bib_file(article_dir, file_name): No Bib files found !
#> BibTeX file does not exist
#> will parse for bibliography
#> bibtex file created
cat(readLines(paste(your_article_path,"example.bib",sep="/")),sep = "\n")
#> @book{ihaka:1996,
#> author = {{Ihaka, Ross and Gentleman, Robert}},
#> title = {{R: A Language for Data Analysis and Graphics}},
#> publisher = {Journal of Computational and Graphical Statistics 3:   :},
#> pages = {299--314},
#> year = {1996},
#> url = {https://doi.org/10.1080/10618600.1996.10474713}
#> }
#> @book{R,
#> author = {R {Core Team}},
#> title = {{R: A Language and Environment for Statistical Computing}},
#> publisher = {R Foundation for Statistical Computing Vienna Austria    :},
#> year = {2016},
#> url = {https://www.R-project.org/},
#> isbn = {3-900051-07-0}
#> }
\end{Sinput}
\end{Schunk}

\subsubsection{Bibliography aggregation}

If you have a BibTeX file and it is missing some references then this mode can help you read the embedded bibliography and combine the two bibliographies in a single BibTeX file.
\begin{verbatim}
# for files with BibTeX source as well as embedded entries
rebib::aggregate_bibliography(your_article_path)
#> Found Bib file  example.bib
#> bibliography aggregation possible
#> Found Bib file  example.bib
#> aggregation delta : 2
#> BibTeX file aggregated
\end{verbatim}


\subsubsection{Using with any LaTeX/.bbl file}

The bibliography converter will convert any LaTeX or \file{.bbl} file given as a path. This way, one can generate BibTeX from the embedded bibliography or \file{.bbl} file.
\begin{verbatim}
rebib::biblio_converter(path_to_your_file)
#> working directory : /temp/Rtmpt6Qk0v/exampledir/standalone
#> bib entries : 6
#> Written BibTeX file : /temp/Rtmpt6Qk0v/exampledir/standalone/sample.bib
\end{verbatim}

\section{Summary}

In summary the \CRANpkg{rebib} package supports:
\begin{itemize}
\item Conversion of embedded natbib type bibliography entries used in the R Journal
\item Aggregation of bibliographic entries from the article as well as separate BibTeX files
\end{itemize}


\address{%
Abhishek Ulayil\\
Student, Institute of Actuaries of India\\%
Mumbai, India\\
ORCiD: 0009-0000-6935-8690\\
}
