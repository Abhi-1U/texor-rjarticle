% !TeX root = RJwrapper.tex
\title{Working with Math environments in texor}
\author{by Abhishek Ulayil}

\maketitle

\abstract{
This is a small sample article to demonstrate usage of texor to convert math environments.
}

\section{Introduction}
Math typesetting has always been LaTeX’s highlight feature, making it a de facto choice among academicians and researchers globally. However, as we proceed to our humble web interfaces, math is hard to describe traditionally. There have been advancements in JavaScript libraries to better Typeset and present math in web pages but not all LaTeX commands/math functions are available.

\subsubsection{MathJax}
The \CRANpkg{texor} package uses Mathjax version 3 to enhance the visual look of the math content in HTML.

The core of the MathJax project is the development of its state-of-the-art, open source, JavaScript platform for display of mathematics. The key design goals are \citep{mathjax}:
\begin{itemize}
    \item High-quality display of mathematics notation in all browsers.
    \item No special browser setup required.
    \item Support for LaTeX, MathML, and other equation markup directly in the HTML source.
    \item An extensible, modular design with a rich API for easy integration into web applications.
    \item Support for accessibility, copy and paste, and other rich functionality.
    \item Interoperability with other applications and math-aware search.
    \item Support for equation conversion outside a browser (e.g., preprocessing on a server).
\end{itemize}

\subsubsection{Pandoc Handling, extensions}
As \CRANpkg{texor} calls \CRANpkg{rmarkdown} to render the Rmarkdwon file into HTML,
the \verb|rjtools::rjournal_web_article| template by default uses MathJax as the math engine. 
Also we specify the Mathjax version in the metadata of the generated Rmarkdown file.

\section{Inline math}
One can define Inline Math in LaTeX using commands \verb|\(..\)| or \verb|$..$|.
It is also possible to use inline math within captions and  
\begin{table}[htbp]
  \centering
  \begin{tabular}{| c | c |}
  \hline
  Command & Math\\
    \hline
    \begin{minipage}{0.45\textwidth}
\vspace{1mm}
\begin{example}
\(\mu = (0,0,0)^\top \)
\end{example}
    \end{minipage} &
    \begin{minipage}{0.45\textwidth}
    \centering
    \( \mu = (0,0,0)^\top \)
    \end{minipage}\\
    \hline
    
  \begin{minipage}{0.45\textwidth}
\vspace{1mm}
\begin{example}
$\mu = (0,0,0)^\top $
\end{example}
    \end{minipage} &
    \begin{minipage}{0.45\textwidth}
    \centering
    $ \mu = (0,0,0)^\top $
    \end{minipage}\\
    \hline
  \end{tabular}
  \caption{InlineMath syntax and its output side by side}
  \label{table:1}
\end{table}



\section{Display Math}



\begin{align}
f(x) = \frac{1}{\sigma\sqrt{2\pi}} 
  \exp\left( -\frac{1}{2}\left(\frac{x-\mu}{\sigma}\right)^{\!2}\,\right)
  \label{eq:2}
\end{align}

\section{Equation Numbering}
\begin{align}
S_{T, s}(z_t) = & X^{\top} K_{b,t}^* X (Z - z_t)^s \nonumber\\
T_{T, s}(z_t) = &X^{\top} K_{b,t}^* Y (Z - z_t)^s,
\label{eq:tvpols}
\end{align}

\section{Unsupported LaTeX Commands}

like instead of \verb|\bm| \verb|\boldmath| use \verb|\mathbb|
avoid custom math commands MathJaX will never be able to parse it


\section{Summary}

In summary the \CRANpkg{texor} package with the help of pandoc and MathJax supports:
\begin{itemize}
\item some common math environments.
\item Equation Numbering
\end{itemize}


\begin{thebibliography}{2}
    \providecommand{\natexlab}[1]{#1}
    \providecommand{\url}[1]{\texttt{#1}}
    \expandafter\ifx\csname urlstyle\endcsname\relax
      \providecommand{\doi}[1]{doi: #1}\else
      \providecommand{\doi}{doi: \begingroup \urlstyle{rm}\Url}\fi

\bibitem[Krewinkel, Lucero (2023)]{pandoc}
A.~ Krewinkel and A.~ Lucero
\newblock pandoc 3.0 Release notes
\newblock \emph{pandoc} \penalty0 2023
\newblock URL \url{https://pandoc.org/releases.html#pandoc-3.0-2023-01-18}

\bibitem[Mathjax authors (2021)]{mathjax}
MathJax authors
\newblock Mathjax website
\newblock  \penalty0 2021
\newblock URL \url{https://www.mathjax.org/#features}

\end{thebibliography}


\address{%
Abhishek Ulayil\\
Student, Institute of Actuaries of India\\%
Mumbai, India\\
ORCiD: 0009-0000-6935-8690\\
}
