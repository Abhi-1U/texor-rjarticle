% !TeX root = RJwrapper.tex
\title{Metadata and Other Structures in texor}
\author{by Abhishek Ulayil}

\maketitle

\abstract{
This is a small sample article to demonstrate handling of metadata and other structures during conversion in texor package.
}

\section{Introduction}
There is a difference in the structure of metadata and generally R Markdown has more
metadata entries than its LaTeX counterpart. So for the conversion, the main task is to 
transform the metadata and include it in the generated R Markdown file, matching the
LaTeX metadata as closely as possible.

When used with the \verb|-s| or \verb|standalone| mode, pandoc reads out some of the existing
meta data. The extracted metadata are listed below.


\subsubsection{Metadata read from LaTeX file}

\begin{itemize}
  \item Author names 
  \item Author address (affiliation)
  \item ORCID number
  \item Title
  \item Bibliography
  \item Abstract
  \item CRAN/Bioconductor packages
\end{itemize}

The \verb|\address{..}| section must be typeset in this manner for the most 
effective conversion.
\begin{verbatim}
\address{
Abhishek Ulayil\\ % Your Name in the first Line
Institute of Actuaries of India (Student)\\% Your affiliation which includes post and/or org
Mumbai, India\\ % Your City/Country
ORCiD: 0009-0000-6935-8690\\% ORCID number with prefix ORCiD: 
}
\end{verbatim}

\subsubsection{Metadata read from DESCRIPTION file}

\begin{itemize}
  \item Date
  \item Slug
  \item Volume and issue
\end{itemize}

Some metadata like the volume and issue were picked up from the folder structure for some legacy articles, where the DESCRIPTION file was not available.

\subsubsection{Other supported commands in conversion}

\begin{table}[htbp]
\centering
\begin{tabular}{| c | c |}
 \hline
 Command & Render \\
 \hline
 \verb|\acronym{A}| & \acronym{A} \\ \hline
\verb|\R| & \R  \\ \hline
 \verb|\pkg{texor}|& \pkg{texor} \\ \hline
 \verb|\CRANpkg{texor}|& \CRANpkg{texor} \\ \hline
 \verb|\BIOpkg{Biobase}|      & \BIOpkg{Biobase}  \\ \hline
 \verb|\ctv{ReproducibleResearch}| & \ctv{ReproducibleResearch}  \\ \hline
\verb|\command{help}| & \command{help} \\
\hline
\end{tabular}
\caption{Other commands supported by \CRANpkg{texor} and pandoc}
\label{table:1}
\end{table}

\section{Miscellaneous}\label{misc}

\subsubsection{Citations}

Table~\ref{table:2} represents all the ways one can use citations in LaTeX and
expect the generated output.

\begin{table}[htbp]
\centering
\begin{tabular}{| c | c |}
 \hline
 Command & Render \\
 \hline
 \verb|\citet{pandoc}| & \citet{pandoc} \\ \hline
\verb|\citep{pandoc}| & \citep{pandoc} \\ \hline
 \verb|\citet*{pandoc}| & \citet*{pandoc} \\ \hline
\verb|\citep*{pandoc}| & \citep*{pandoc} \\ \hline
 \verb|\citeyear{pandoc}| & \citeyear{pandoc} \\ \hline
\verb|\citeauthor{pandoc}| & \citeauthor{pandoc} \\ \hline
 \verb|\cite{pandoc}| & \cite{pandoc} \\ \hline
\end{tabular}
\caption{Different forms of citation in LaTeX.}
\label{table:2}
\end{table}

\subsubsection{Footnotes}

Footnotes do not work in \verb|tabular| environments in LaTeX but do work in R Markdown tables\footnote{Footnotes in paragraph text work in both LaTeX and R markdown}. Custom footnote numbering works in LaTeX but does not work in R Markdown\footnote[25]{Footnote 25 }. These features are demonstrated in Table~\ref{table:3}.

\begin{table}[htbp]
\centering
\begin{tabular}{| c | c |}
 \hline
 Command & Render \\
 \hline
 \verb|\acronym{A}\footnote{This works in HTML, but not in LaTeX}| & \acronym{A}\footnote{This works in HTML, but not in LaTeX}} \\ \hline
\verb|\R\footnote[25]{HTML shows 4 rather than 25}| & \R\footnote[25]{HTML shows 4 rather than 25}  \\ \hline
\end{tabular}
\caption{Demonstration of footnotes.}
\label{table:3}
\end{table}


\subsubsection{Block quotes}
\begin{quote}
This block quote should justify an example.
\end{quote}

\subsubsection{Links}
Most links work just fine using \verb|\href| and the \verb|\url| command will work flawlessly, as demonstrated in Table ~\ref{table:4}.

\begin{table}[htbp]
\centering
\begin{tabular}{| c | c |}
 \hline
 Command & Render \\
 \hline
 \verb|\href{www.google.com}{Google}| & \href{www.google.com}{Google} \\ \hline
\verb|\url{www.google.com}| & \url{www.google.com}  \\ \hline
\verb|Table \ref{table:3}| & Table \ref{table:3}  \\ \hline
\verb|Section \nameref{misc}| & Section \nameref{misc}\\ \hline
\verb|\autoref{table:3}| & \autoref{table:3} \\ \hline
\end{tabular}
\caption{Links}
\label{table:4}
\end{table}

\paragraph{Note}
Certain reference commands like \verb|\nameref{}| and \verb|\autoref{}| may not work as intended in HTML.

\subsubsection{Lists}

An example of usage of lists in LaTeX and \pkg{texor} generated HTML.

\paragraph{itemize}
\begin{itemize}
\item A
\item B
\end{itemize}
\paragraph{enumerate}
\begin{enumerate}
\item A
\item B
\end{enumerate}
\paragraph{description}
\begin{description}
\item[\pkg{rebib}] Convert and Aggregate Bibliographies
\item[\pkg{texor}] Converting 'LaTeX' 'R Journal' Articles into 'RJ-web-articles'
\end{description}

\paragraph{Note:}
Some options for lists such as special numbering schemes or markups to description
item points might not work as intended and the generated HTML might be simple.

\subsubsection{Colored text}
Some commands to describe colored text also work during conversion, such as
\verb|\textcolor{blue}{Blue Text}|  \textcolor{blue}{Blue Text}.

\section{Theorem environments}

\begin{theorem}\label{theorem:1}
This is a sample theorem..
\end{theorem}

\begin{proposition}\label{prop:1}
As with $X \in \mathbb{R}$ this proposition ...  
\end{proposition}

\begin{remark}\label{rm:1}
\CRANpkg{texor} makes the article conversion easier.
\end{remark}

\pkg{Pandoc} supports default LaTeX theorem environments. One can refer to these environments
like Theorem~\ref{theorem:1},Proposition~\ref{prop:1} and Remark~\ref{rm:1} made here.
However custom theorem environments are not supported yet.

\section{Summary}
In summary the \CRANpkg{texor} package alongside pandoc supports:
\begin{itemize}
\item Extracting metadata and putting it together well.
\item Handling links, block quotes, citations and footnotes.
\item Default theorem environments.
\end{itemize}

\begin{thebibliography}{2}
    \providecommand{\natexlab}[1]{#1}
    \providecommand{\url}[1]{\texttt{#1}}
    \expandafter\ifx\csname urlstyle\endcsname\relax
      \providecommand{\doi}[1]{doi: #1}\else
      \providecommand{\doi}{doi: \begingroup \urlstyle{rm}\Url}\fi

\bibitem[Krewinkel, Lucero (2023)]{pandoc}
A.~ Krewinkel and A.~ Lucero
\newblock pandoc 3.0 Release notes
\newblock \emph{pandoc} \penalty0 2023
\newblock URL \url{https://pandoc.org/releases.html}

\bibitem[Fairbairns, Duggan, Schöpf, Eijkhout (2011)]{moreverb}
R.~ Fairbairns, A.~ Duggan, R.~ Schöpf and V.~ Eijkhout
\newblock The moreverb package documentation
\newblock \emph{CTAN} \penalty0 2011
\newblock URL \url{https://mirror.niser.ac.in/ctan/macros/latex/contrib/moreverb/moreverb.pdf}

\end{thebibliography}


\address{%
Abhishek Ulayil\\
Institute of Actuaries of India (Student)\\%
Mumbai, India\\
ORCiD: 0009-0000-6935-8690\\
}
