% !TeX root = RJwrapper.tex
\title{Metadata and Other Structures in texor}
\author{by Abhishek Ulayil}

\maketitle

\abstract{
This is a small sample article to demonstrate handling of metadata and other structures during conversion in texor package.
}

\section{Introduction}
There is a difference in the structure of metadata and generally Rmarkdown has more
meta-data entries than its LaTeX counterpart. So for the conversion, main task is to 
transform the meta data and include it in the generated Rmarkdown file, matching the
LaTeX data as closely as possible.

Pandoc when used with \verb|-s| or \verb|standalone| mode reads out some of the existing
meta data. The extracted author meta-data are listed below:


\subsubsection{Metadata Read from LaTeX file}

\begin{itemize}
  \item Author Names 
  \item Author address(affiliation)
  \item ORCID number
  \item Title
  \item Bibliography
  \item Abstract
  \item CRAN/BIO packages
\end{itemize}

The \verb|\address{..}| section must be typeset in this manner for the most 
effective conversion.
\begin{verbatim}
\address{
Abhishek Ulayil\\ % Your Name in the first Line
Institute of Actuaries of India (Student)\\% Your affiliation which includes post and/or org
Mumbai, India\\ % Your City/Country
ORCiD: 0009-0000-6935-8690\\% ORCID number with prefix ORCiD: 
}
\end{verbatim}

\subsubsection{Metadata Read from DESCRIPTION file}

\begin{itemize}
  \item Date
  \item Slug
  \item Volume and Issue
\end{itemize}

Some details like the Volume and Issue were picked up from folder structure as well
in some legacy articles where DESCRIPTION file was not available.

\subsubsection{Other supported commands in conversion}

\begin{table}[htbp]
\centering
\begin{tabular}{| c | c |}
 \hline
 Command & Render \\
 \hline
 \verb|\acronym{A}| & \acronym{A} \\ \hline
\verb|\R| & \R  \\ \hline
 \verb|\pkg{texor}|& \pkg{texor} \\ \hline
 \verb|\CRANpkg{texor}|& \CRANpkg{texor} \\ \hline
 \verb|\BIOpkg{Biobase}|      & \BIOpkg{Biobase}  \\ \hline
 \verb|\ctv{ReproducibleResearch}| & \ctv{ReproducibleResearch}  \\ \hline
\verb|\command{help}| & \command{help} \\
\hline
\end{tabular}
\caption{Other commands supported by \CRANpkg{texor} and Pandoc}
\label{table:1}
\end{table}

\section{Miscellaneous}

\subsubsection{Citations}

Table \ref{table:2} represents all the ways one can use citations in LaTeX and
expect the generated output.

\begin{table}[htbp]
\centering
\begin{tabular}{| c | c |}
 \hline
 Command & Render \\
 \hline
 \verb|\citet{pandoc}| & \citet{pandoc} \\ \hline
\verb|\citep{pandoc}| & \citep{pandoc} \\ \hline
 \verb|\citet*{pandoc}| & \citet*{pandoc} \\ \hline
\verb|\citep*{pandoc}| & \citep*{pandoc} \\ \hline
 \verb|\citeyear{pandoc}| & \citeyear{pandoc} \\ \hline
\verb|\citeauthor{pandoc}| & \citeauthor{pandoc} \\ \hline
 \verb|\cite{pandoc}| & \cite{pandoc} \\ \hline
\end{tabular}
\caption{Different citations in LaTeX}
\label{table:2}
\end{table}

\subsubsection{Footnotes}
\begin{table}[htbp]
\centering
\begin{tabular}{| c | c |}
 \hline
 Command & Render \\
 \hline
 \verb|\acronym{A}\footnote{This works in Web article}| & \acronym{A}\footnote{This works in Web article} \\ \hline
\verb|\R\footnote[21]{Its 2 not 21 :(}| & \R\footnote[21]{Its 2 not 21 :(}  \\ \hline
\end{tabular}
\caption{Demo of footnotes}
\label{table:3}
\end{table}

Footnotes do not work in \verb|tabular| environment in LaTeX but it does work in Rmarkdown\footnote{This works in LaTeX and  Web article}.
Custom Footnote number works in LaTeX but does not work in web articles\footnote[21]{Forever 21 :)}.

\subsubsection{BlockQuotes}
\begin{quote}
This BlockQuote should Justify an example.
\end{quote}

\subsubsection{Links}
Most links work just fine using \verb|\href| and \verb|\url| command will work flawlessly.
\begin{table}[htbp]
\centering
\begin{tabular}{| c | c |}
 \hline
 Command & Render \\
 \hline
 \verb|\href{www.google.com}{Google}| & \href{www.google.com}{Google} \\ \hline
\verb|\url{www.google.com}| & \url{www.google.com}  \\ \hline
\verb|Table \ref{table:3}| & Table \ref{table:3}  \\ \hline
\end{tabular}
\caption{Links}
\label{table:4}
\end{table}

\subsection{Colored text}
Some commands to describe colored text also work during conversion like
\verb|\textcolor{blue}{Blue Text}|  \textcolor{blue}{Blue Text}

\section{Summary}
In summary the \CRANpkg{texor} package alongside pandoc supports:
\begin{itemize}
\item Extracting Metadata and putting it together well.
\item Handling Links, BlockQuotes, Citations and Footnotes.
\end{itemize}

\begin{thebibliography}{2}
    \providecommand{\natexlab}[1]{#1}
    \providecommand{\url}[1]{\texttt{#1}}
    \expandafter\ifx\csname urlstyle\endcsname\relax
      \providecommand{\doi}[1]{doi: #1}\else
      \providecommand{\doi}{doi: \begingroup \urlstyle{rm}\Url}\fi

\bibitem[Krewinkel, Lucero (2023)]{pandoc}
A.~ Krewinkel and A.~ Lucero
\newblock pandoc 3.0 Release notes
\newblock \emph{pandoc} \penalty0 2023
\newblock URL \url{https://pandoc.org/releases.html}

\bibitem[Fairbairns, Duggan, Schöpf, Eijkhout (2011)]{moreverb}
R.~ Fairbairns, A.~ Duggan, R.~ Schöpf and V.~ Eijkhout
\newblock The moreverb package documentation
\newblock \emph{CTAN} \penalty0 2011
\newblock URL \url{https://mirror.niser.ac.in/ctan/macros/latex/contrib/moreverb/moreverb.pdf}

\end{thebibliography}


\address{%
Abhishek Ulayil\\
Institute of Actuaries of India (Student)\\%
Mumbai, India\\
ORCiD: 0009-0000-6935-8690\\
}
