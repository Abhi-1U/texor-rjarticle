% !TeX root = RJwrapper.tex
\title{Working with Code environments in texor}
\author{by Abhishek Ulayil}

\maketitle

\abstract{
This is a small sample article to demonstrate usage of texor to convert code environments.
}

\section{Introduction}

Pandoc naturally converts verbatim environment easily, however the redefination of other commands such as example, example*, Sinput etc to verbatim does not work well in pandoc.

Hence Texor package uses the stream editor to search find and replace matching code environments to verbatim before pandoc touches it.

This way the the code is not lost in conversion, also a pandoc extension is used to add attributes to the markdown code using 
'fenced\_code\_attributes' 

Code Environment types are well summarized in the table \ref{table:1}

\begin{table}[htbp]
\centering
\begin{tabular}{l | lllll }
 \hline
 Code Environment Type &  &  &  & & \\
 \hline
 Example          & example & example* & smallexample & & \\
 S.series         & Sin & Sout & Sinput & Soutput & Scode \\
 Special Verbatim & smallverbatim & boxedverbatim &  & & \\
\hline
\end{tabular}
\caption{Code Environment support in texor}
\label{table:1}
\end{table}


\section{Environments}

\subsection{Verbatim Series}
While verbatim is naturally supported in pandoc,other extensions of verbatim environment
like boxedverbatim from moreverb package \citep{moreverb}.
% verbatim

1. verbatim :

\begin{verbatim}
print("Hello world")
\end{verbatim}


2. smallverbatim :

\begin{smallverbatim}
print("Hello world")
\end{smallverbatim}


3. boxedverbatim :

\begin{boxedverbatim}
print("Hello world")
\end{boxedverbatim}





\subsection{S series}

S series code environement is defined in Rjournal.sty file.

1. Sinput :

\begin{Sinput}
print("Hello world")
\end{Sinput}


2. Soutput :

\begin{Soutput}
[1] "hello world"
\end{Soutput}


3. Sin :

\begin{Sin}
print("Hello world")
\end{Sin}


4. Sout :
\begin{Sout}
[1] "hello world"
\end{Sout}


\subsection{Example series}
Example series of code environment is defined in Rjournal.sty file.

1. example :

\begin{example}
print("Hello world")
\end{example}


2. example* :

\begin{example*}
print("Hello world")
\end{example*}


3. smallexample :

\begin{smallexample}
print("Hello world")
\end{smallexample}

\section{Code in Figure Environments}
A small example of this is visble in \ref{code:example}

\begin{figure*}[htbp]
\begin{Sinput}
code_in_figure <- function() {
  if (pandoc_version >= 3) {
    print("Code in Figure Supported")
  }
  else {
    print("code in Figure not supported")
  }
}
\end{Sinput}
\caption{ Example Code inside Figure environment}
\label{code:example}
\end{figure*}
Pandoc v3 or greater \citep{pandoc} has a Figure object which allows non-image
figures to be treated like one.

\section{Code in Table Environments}

\section{Inline Code usage}


\section{Code chunks using Schunk}

\begin{Schunk}
Input :
\begin{Sinput}
print("Hello world")
\end{Sinput}
Output :
\begin{Soutput}
[1] "hello world"
\end{Soutput}

\end{Schunk}
\section{Summary}

In summary the \CRANpkg{texor} package supports:
\begin{itemize}
\item Almost all code environments in RJournal.
\item Code Highlight for R language.
\item Code blocks inside Figure environment with captions.
\item Inline Code.
\end{itemize}



\begin{thebibliography}{2}
    \providecommand{\natexlab}[1]{#1}
    \providecommand{\url}[1]{\texttt{#1}}
    \expandafter\ifx\csname urlstyle\endcsname\relax
      \providecommand{\doi}[1]{doi: #1}\else
      \providecommand{\doi}{doi: \begingroup \urlstyle{rm}\Url}\fi

\bibitem[Krewinkel, Lucero (2023)]{pandoc}
A.~ Krewinkel and A.~ Lucero
\newblock pandoc 3.0 Release notes
\newblock \emph{pandoc} \penalty0 2023
\newblock URL \url{https://pandoc.org/releases.html#pandoc-3.0-2023-01-18}

\bibitem[Fairbairns, Duggan, Schöpf, Eijkhout (2011)]{moreverb}
R.~ Fairbairns, A.~ Duggan, R.~ Schöpf and V.~ Eijkhout
\newblock The moreverb package documentation
\newblock \emph{CTAN} \penalty0 2011
\newblock URL \url{https://mirror.niser.ac.in/ctan/macros/latex/contrib/moreverb/moreverb.pdf}
\end{thebibliography}


\address{%
Abhishek Ulayil\\
Student, Institute of Actuaries of India\\%
Mumbai, India\\
ORCiD: 0009-0000-6935-8690\\
}
